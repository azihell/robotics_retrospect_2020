\chapter{Resultados}
\label{chap:result}

Nesta seção serão demonstrados os métodos e resultados dos artigos provenientes dos
trabalhos desenvolvidos durante o período do curso de formação em Robótica e Sistemas Autônomos. Nos Apêndices estão identificados os artigos publicados e certificados obtidos.

%--------- NEW SECTION ----------------------
\section{Artigo SAPCT}
\label{sec:artigoRAJA}

Para o evento V Seminário de Avaliação de Pesquisa Científica e Tecnológica (SAPCT)
foi realizado o artigo Projeto e Simulação de um Manipulador Robótico com 5 Graus de
Liberdade e Sistema de Visão Integrado, com base no projeto do manipulador robótico
RAJA, este artigo consta no Apêndice \ref{app:rajapaper}. O trabalho apresentado foi premiado como melhor trabalho na categoria de bolsistas de projeto PD\&I.


\section{Integração do sistema}
\label{sec:artigoTRIS}

Esse artigo refere-se ao projeto ``TRIS'' que foi criado a partir da necessidade gerada pela pandemia do COVID-19 de isolar possíveis vetores do vírus. Seu funcionamento é dado pela identificação de pessoas e avaliação das suas temperaturas através de câmeras (RGB e espectro infravermelho). Os algoritmos requerem um computador treinado através de uma rede neural no reconhecimento de rostos e aferição da temperatura. Caso valores acima de 37,8$\circ$C sejam detectados, o programa irá alertar um operador diante do sistema para que este informasse à pessoa sobre a possibilidade dela estar num ou estado febril, que é um dos sintomas do COVID-19. Esse sistema foi criado com o próposito de auxiliar no controle da propagação do vírus. Nesse projeto puderam ser desenvolvidos os conhecimentos de rede neural, interface de sistemas, concepção de um banco de dados, utilização de câmeras RGB e espectro infravermelho. O resultado obtido do projeto do TRIS foi o artigo publicado no evento VI International Symposium on Innovation and Technology (SIINTEC) 2020, e posteriormente sua premiação em primeiro lugar dentre os trabalhos apresentados. Este artigo e o certificado de participação nestes evento constam, respectivamente, nos Apêndices \ref{app:trispaper} e \ref{app:certif}



