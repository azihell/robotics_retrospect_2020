\chapter{Conclusão}
\label{chap:conc}

A especialização em Robótica e Sistemas autônomos proporcionada pelo Laboratório de Robótica e Sistemas Autônomos do SENAI CIMATEC extensa e abrangente, uma convergência de métodos, técnicas e conhecimentos de diversas áreas, como naturalmente acontece em pesquisas em Robótica, um campo de incontáveis possibilidades. O Laboratório é fruto dos esforços de pesquisadores já experientes, e a sua intenção de capacitar futuros companheiros de trabalho foi o que gerou o programa de capacitação ``Novos Talentos'', o qual fiz parte juntamente com outros 11 colegas.

As diversas etapas presentes na condução de projetos de pesquisa em Robótica foram trazidas nas formas dos Desafios propostos nas seções expostas no Capítulo \ref{chap:desenvolvimento}, e analisar estes documentos revela a amplitude de tarefas e conhecimento necessários para o sucesso em um projeto. Não tão transparente, todavia, está o aparato e os métodos utilizados em gerência de projetos. Este tema foi amplamente explorado embora uma documentação expressiva não tenha sido gerada. 
