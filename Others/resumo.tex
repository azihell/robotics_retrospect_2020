\begin{thesisresumo}

  Este relatório visa exibir de forma resumida os resultados dos projetos executados no curso de formação em Robótica e Sistemas Autônomos. O projeto abraçou 12 graduados nas áreas de Engenharia Elétrica, Mecânica, da Computação e de Automação e Controle e os treinou no uso de ferramentas CAD na modelagem de objetos, ferramentas de simulação, de gerenciamento de projetos, de desenvolvimento e versionamento de códigos nas linguagens Python, C++ (especialmente para ROS) e R. Essas ferramentas foram gradualmente aprendidas para que pudessem ser utilizada ao longo do programa de formação na confecção um manipulador robótico e na configuração uma plataforma móvel para realizar mapeamento e navegação autônomas, além da manipulação. Os projetos foram organizados de modo a alocar as capacidades dos graduados em equipes contendo diversidades de conhecimentos, havendo sempre a figura de um líder de projetos.

\ \\

% use de três a cinco palavras-chave

\textbf{Palavras-chave}: Robótica, Manipuladores,ROS, Linguagens de programação, Estatística

\end{thesisresumo}